% Created 2015-05-04 Mon 22:46
\documentclass[11pt]{article}
\usepackage[utf8]{inputenc}
\usepackage[T1]{fontenc}
\usepackage{fixltx2e}
\usepackage{graphicx}
\usepackage{longtable}
\usepackage{float}
\usepackage{wrapfig}
\usepackage{rotating}
\usepackage[normalem]{ulem}
\usepackage{amsmath}
\usepackage{textcomp}
\usepackage{marvosym}
\usepackage{wasysym}
\usepackage{amssymb}
\usepackage{hyperref}
\tolerance=1000
\usepackage[utf8]{inputenc}
\usepackage[usenames,dvipsnames]{color}
\usepackage{commath}
\usepackage{tikz}
\usetikzlibrary{shapes,backgrounds}
\usepackage{marginnote}
\usepackage{listings}
\usepackage{color}
\usepackage{enumerate}
\hypersetup{urlcolor=blue}
\hypersetup{colorlinks,urlcolor=blue}
\setlength{\parskip}{16pt plus 2pt minus 2pt}
\definecolor{codebg}{rgb}{0.96,0.99,0.8}
\definecolor{codestr}{rgb}{0.46,0.09,0.2}
\author{Oleg Sivokon}
\date{\textit{<2015-04-03 Fri>}}
\title{Assignment 13, Infinitesimal Calculus}
\hypersetup{
  pdfkeywords={Infinitesimal Calculus, Assignment, Definition of Limits},
  pdfsubject={Third asssignment in the course Infinitesimal Calculus},
  pdfcreator={Emacs 25.0.50.1 (Org mode 8.2.2)}}
\begin{document}

\maketitle
\tableofcontents


\lstset{ %
  backgroundcolor=\color{codebg},
  basicstyle=\ttfamily\scriptsize,
  breakatwhitespace=false,         % sets if automatic breaks should only happen at whitespace
  breaklines=false,
  captionpos=b,                    % sets the caption-position to bottom
  commentstyle=\color{mygreen},    % comment style
  framexleftmargin=10pt,
  xleftmargin=10pt,
  framerule=0pt,
  frame=tb,                        % adds a frame around the code
  keepspaces=true,                 % keeps spaces in text, useful for keeping indentation of code (possibly needs columns=flexible)
  keywordstyle=\color{blue},       % keyword style
  showspaces=false,                % show spaces everywhere adding particular underscores; it overrides 'showstringspaces'
  showstringspaces=false,          % underline spaces within strings only
  showtabs=false,                  % show tabs within strings adding particular underscores
  stringstyle=\color{codestr},     % string literal style
  tabsize=2,                       % sets default tabsize to 2 spaces
}

\clearpage

\section{Problems}
\label{sec-1}

\subsection{Problem 1}
\label{sec-1-1}
Given the sequence $(a_n)$ defined as $0 < a_1 < 6$ and $a_{n+1} = \sqrt{6a_n}$
for all $n$.
\begin{enumerate}
\item Prove that the sequence converges.
\item Find $\lim_{n \to \infty} a_n$.
\end{enumerate}

\subsubsection{Answer 1}
\label{sec-1-1-1}
First, observe that every next element of the sequence $(a_n)$ is bigger than
the previous.  Using mathematical induction we can show that:

\textbf{Base step}: $\sqrt{a_1} < \sqrt{a_2}$ because $a_1 < 6$ and thus
$\sqrt{a_1} < \sqrt{6}$, consequently (from order preservation under multiplication)
$\sqrt{a_1} \times \sqrt{a_1} < \sqrt{6} \times \sqrt{a_1}$ and hence
$\sqrt{a_1} < \sqrt{a_2}$.

\textbf{Inductive step}:
\begin{equation*}
  \begin{aligned}
    a_k                        &< \sqrt{6 \times a_{k+1}} \\
                               &\textit{Using induction hypothesis, $k = n - 1$} \\
    \sqrt{6 \times a_k}        &< \sqrt{6 \times a_{k+1}} \\
    \sqrt{6} \times \sqrt{a_k} &< \sqrt{6} \times \sqrt{a_{k+1}} \\
                               &\textit{Order preservation under multiplication} \\
    \sqrt{a_k}                 &< \sqrt{a_{k+1}} \\
    a_k                 &< a_{k+1}
  \end{aligned}
\end{equation*}

Hence, by mathematical induction, the sequence $(a_n)$ is an increasing one.

Now we demonstrate that this sequence is bounded above and it has a supremum.
Again, using induction, we can choose 6 to be the upper bound.

\textbf{Base step}: $a_1 < 6$ by definition.

\textbf{Inductive step}:
\begin{equation*}
  \begin{aligned}
    a_k                 &< 6 \\
                        &\textit{Using induction hypothesis, $k = n - 1$} \\
    \sqrt{6 \times a_k} &< \sqrt{6 \times 6} \\
    \sqrt{6} \times \sqrt{a_k} &< \sqrt{6} \times \sqrt{6} \\
                        &\textit{Order preservation under multiplication} \\
    \sqrt{a_k}          &< \sqrt{6} \\
    a_k &< 6
  \end{aligned}
\end{equation*}

Hence, by mathematical induction, $(a_n)$ has an upper bound. It is trivial
to see that this is also the supremum, since $a_1$ is defined to be strictly
less than 6, so no smaller upper bound can exist.

Having showed both these properties, we can now use the monotone sequence
theorem to claim that this sequence converges.  Accidentally, we also found
the limit of this sequence, which wasn't due until the next question.
\subsubsection{Answer 2}
\label{sec-1-1-2}
We have already found the limit of $(a_n)$ as $n$ approaches infinity.  This
is given by a corollary from the monotone sequence theorem: the supremum
is also the limit of a monotone sequence.  Thus the answer is 6.
\subsection{Problem 2}
\label{sec-1-2}
Find the limits of:

\begin{enumerate}
\item \begin{equation*}
  \lim_{n \to \infty} \frac{(-5)^n - 2^n + 2}{3^n + (-2)^n - 2}.
\end{equation*}

\item \begin{equation*}
  \lim_{n \to \infty} \frac{3^n + (-2)^n - 2}{(-5)^n - 2^n + 2}.
\end{equation*}

\item \begin{equation*}
  \lim_{n \to \infty} (\lfloor 2n \rfloor - 2 \lfloor n \rfloor).
\end{equation*}

\item \begin{equation*}
  \lim_{n \to \infty} \frac{\sqrt[n]{n!}}{n}.
\end{equation*}
\end{enumerate}

\subsubsection{Answer 3}
\label{sec-1-2-1}
We can see that the denominator of this limit will be a positive
integer for $n > 1$.  The numerator will alternate between positive
and negative values, where the difference will only increase with
time since $(-5)^2n > 2^n$ and $(-5)^{2n-1} > 2^n$, thus this expression
has two limit points: $\infty$ and $-\infty$.
\subsubsection{Answer 4}
\label{sec-1-2-2}
The same reasoning we put forward in \ref{sec-1-2-1} applies here, but now the
numerator will be some positive integer, smaller than the denominator.
Thus this expression will have a single limit at 0.
\subsubsection{Answer 5}
\label{sec-1-2-3}
Here I will assume $n$ to be real, otherwise the limit is trivially 0.
In this case we can split the sequence into the following subsequences:
\begin{equation*}
  \begin{aligned}
    (a_n) = \{\lfloor x \rfloor | x \in \mathbb{R}: x \pmod 2 < 0.5 \} \\
    (b_n) = \{\lfloor x \rfloor | x \in \mathbb{R}: x \pmod 2 \geq 0.5 \}
  \end{aligned}
\end{equation*}

It's easy to see that $(a_n) \cup (b_n)$ gives the initial set.  These
subsets have constant limits $\{0, 1\}$ as is given by their generating
formulae: $\lfloor x \pmod 2 < 0.5 \rfloor$ and $\lfloor x \pmod 2 < 0.5
    \rfloor$ resp.  But the whole expression doesn't have a single limit point,
since we can always find as many points as we like that would not be an
$\epsilon$ distance from one of the limit points found.
\subsubsection{Answer 6}
\label{sec-1-2-4}
We can show this expression approaches 1 as $n$ gets larger using the
following algebraic transformations:
\begin{equation*}
  \begin{aligned}
    \frac{\sqrt[n]{n!}}{n} &= \sqrt[n]{\frac{n!}{n^n}} \\
                           &= \sqrt[n]{\prod_{k=1}^n\frac{n - k}{n}}
  \end{aligned}
\end{equation*}

Since $n - k < n$, $\frac{n - k}{n} < 1$.  The product of arbitrary many
terms all of which are less than one will be less than one.  And will
become smaller as the number of terms grows.  However, the nth root
will become progressively large as the fractional term becomes smaller,
yet never exceeding 1.  Thus the limit of this expression is 1.
\subsection{Problem 3}
\label{sec-1-3}
Let $(a_n)$ and $(b_n)$ be sequences both bounded from above.
\begin{enumerate}
\item Prove $\sup \{a_n + b_n | n \in \mathbb{N}\} \leq \sup
      \{a_n | n \in \mathbb{N}\} + \sup \{b_n | n \in \mathbb{N}\}$.
\item Find such $(a_n)$ and $(b_n)$ which are equal under the condition
given in (1).
\item Find such $(a_n)$ and $(b_n)$ for which the inequality given in
(1) holds.
\end{enumerate}

\subsubsection{Answer 7}
\label{sec-1-3-1}
Suppose, for contradiction, that there is some such sequences $(a_n)$ and
$(b_n)$, for which $\sup \{a_n + b_n | n \in \mathbb{N}\} > \sup \{a_n | n
    \in \mathbb{N}\} + \sup \{b_n | n \in \mathbb{N}\}$.  But we know that every
element of $(a_n)$ is at least as large as $\sup (a_n)$, and the same is
true of $(b_n)$.  Let's call the supremum of the sum of the sequences $s_1$,
the supremum of $(a_n)$---$s_2$ and the supremum of $(b_n)$--- $s_3$.
Since $s_1 > s_2 + s_3$, either $(a_n)$ or $(b_n)$ must contain an element
which is as large as $s_2 + s_1 - s_2$ (and similarly for $(b_n)$.)  But
this is a contradiction, since $s_2$ is the largest element of $(a_n)$
(and similarly for $(b_n)$.)  Hence, the initial assumption is false, hence
the supremum of the element-wise sum of two sequences cannot be greater than
the sum of the suprema of these sequences.
\subsubsection{Answer 8}
\label{sec-1-3-2}
An example of $(a_n) = \frac{1}{n}$ whose supremum is 1 and $(b_n) =
    \frac{2}{n}$.  Since they both reach their maximum output at $n=1$, the sum
of their suprema and the supremum of their sum is the same.
\subsubsection{Answer 9}
\label{sec-1-3-3}
To contrast \ref{sec-1-3-2}, $(a_n) = \frac{1}{n}$ and $(b_n) = \frac{1}{n - 1}$
reach their suprema at different times, ($(b_n)$ isn't even defined at the
time $(a_n)$ reaches its maximum), the supremum of their sum is thus 1.5,
but the sum of their suprema is 2.
\subsection{Problem 4}
\label{sec-1-4}
Let $(a_n) = n - \lfloor \sqrt{n} \rfloor^2$.
\begin{enumerate}
\item Prove that $(a_n)$ is bounded from below.
\item Prove that 0 is a limit point of $(a_n)$.
\item Find $\inf \{a_n | n \in \mathbb{N}\}$, $\liminf_{n \to \infty} a_n$.
Establish whether $(a_n)$ has minima.
\item Given natural number $\ell$, prove that almost for all $n$ it holds
that $n < \sqrt{n^2 + \ell} < n + 1$.
\item Prove that every natural number is a limit point of $(a_n)$.
\item Is $(a_n)$ bounded from above?
\item Find $\limsup_{n \to \infty} a_n$.
\end{enumerate}

\subsubsection{Answer 10}
\label{sec-1-4-1}
I will claim that the lower bound on this sequence is 0.  The term
$\lfloor \sqrt{n} \rfloor^2$  can be either equal to $n$ (when $n$ is
a perfect sqare), or it can be smaller than $n$.  Assume, for contradiction,
that $\lfloor \sqrt{n} \rfloor^2 > n$, then it gives
$\lfloor \sqrt{n} \rfloor > \sqrt{n}$, but since $n$ is positive, this is
impossible.  Hence contradiction.  Hence, the original claim stands.
\subsubsection{Answer 11}
\label{sec-1-4-2}
We can choose the subsequence of $(a_n)$ such that it consists only
of the perfect squares.  In which case the $\lim_{m \to \infty}(a_m) = 0$
because $m = n * n$, and subsequently $m = \lfloor \sqrt{m} \rfloor^2$.
\subsubsection{Answer 12}
\label{sec-1-4-3}
Following the argument similar to \ref{sec-1-4-1}, the greatest lower bound on
$(a_n)$ is 0.  Assume there was some positive $x$, a candidate for a greater
lower bound, then $n - \lfloor \sqrt{n} \rfloor^2 > x$.  Let's try $n = 1$
to show the contradiction.  Substituting $n = 1$ gives $0 > x$, but we
assumed $x > 0$.  Hence, contradiction, hence the sequence has greates
lower bound at zero.
\subsubsection{Answer 13}
\label{sec-1-4-4}
Through some algebraic transformations we arrive at:
\begin{flalign*}
  &n < \sqrt{n^2 + \ell} < n + 1 \\
  &n^2 < n^2 + \ell < n^2 + 2n + 1 \\
  &\textit{First inequality is obviously true for $\ell > 0$} \\
  &n^2 + \ell < n^2 + 2n + 1 \\
  &\ell < 2n + 1 \\
  &\textit{Second inequality is true for $\ell < 2n + 1$}.
\end{flalign*}

Thus the left side of inequality holds for $\ell > 0$ and the right
side holds for $\ell < 2n + 1$.  Since $\ell$ is a constant, and $n$
can be made as large as desired, the inequality holds for almost all $n$.
\subsubsection{Answer 14}
\label{sec-1-4-5}
Pick a natural number $x$, without loss of generality we assume:
$\lim_{m \to \infty}(a_m) = x$, where $m$ is somehow choosen from the
set of all values of $n$.  We can define a selection rule to be
``whenever $n + x$ is a perfect square'', since there are infinitely
many perfect squares, and there isn't a largest perfect square, $x$
can be any number.
\subsubsection{Anser 15}
\label{sec-1-4-6}
No, $(a_n)$ doesn't have a supermum.  Assume it had one, let's say $N$, then
for all $n - \lfloor \sqrt{n} \rfloor^2 < N$.  But we could think of $m =
    N^2 - 1$, which if we substitute back into our original formula:
\begin{flalign*}
  &m - \lfloor \sqrt{m} \rfloor^2 < N \\
  &N^2 - 1 - \lfloor \sqrt{N^2 - 1} \rfloor^2 < N \\
  &N^2 - 1 - \sqrt{(N - 1)^2}^2 < N \\
  &\textit{$N^2 - 1$ cannot be a perfect square} \\
  &N^2 - 1 - (N - 1)^2 < N \\
  &N^2 - 1 - N^2 + 2N - 1 < N \\
  &2N - 2 < N.
\end{flalign*}

Which is a contradiction (because $N$ must be greater than 2).  Hence no
supremum exists.
\subsubsection{Answer 16}
\label{sec-1-4-7}
Since we just showed that $(a_n)$ diverges, the $\limsup_{n \to \infty}(a_n)$
doesn't exist in the narrow sense of the word, or is equal to $\infty$ in a
more general sense.
\subsection{Problem 5}
\label{sec-1-5}
Let $(a_n)$ and $(b_n)$ be sequences.
\begin{enumerate}
\item Assume $\lim_{n \to \infty}(a_n + b_n) = L$ for some finite $L$.
Prove that if $(a_n)$ is bounded then $(b_n)$ is bounded too.
\item Assume $\lim_{n \to \infty}(a_n + b_n) = L$ for some finite $L$.
Prove that if $(a_n)$ has a limit point $a$, then $L - a$ is a
limit point of $(b_n)$.
\item Assume $(a_n)$ has 20106 limit points and $(b_n)$ has 20474 limit
points. Prove that $(a_n + b_n)$ diverges.
\end{enumerate}

\subsubsection{Answer 17}
\label{sec-1-5-1}
Assume, for contradiction, $(b_n)$ is unbounded, then $\lim_{n \to
    \infty}(b_n)$ is either $\infty$ or $-\infty$, but from the addition
rules involving infinity we know that $\infty + L = \infty$ and
$L - \infty = -\infty$, but we are given that the sequence converges,
hence contradiction.  Hence $(b_n)$ must be bounded.
\subsubsection{Answer 18}
\label{sec-1-5-2}
Since the sequence of sum converges, each one of its subsequences converges
to the same limit.  Hence both $(a_n)$ and $(b_n)$ have defined limits.  We
can use the limit sum law to show that $\lim_{n \to \infty}(a_n + b_n) =
    \lim_{n \to \infty}(a_n) + \lim_{n \to \infty}(b_n) = L$.  Since we are
given $\lim_{n \to \infty}(a_n) = a$, it follows that $\lim_{n \to
    \infty}(b_n) = L - a$, which completes the proof.
\subsubsection{Answer 19}
\label{sec-1-5-3}
By pigeonhole principle, there must be a limit point exclusive to $(b_n)$.
Existence of this limit point would violate what we proved in \ref{sec-1-5-1}.
Since the statement in previous answer is a biconditional, then, in
particular, it follows that sequence $(a_n + b_n)$ diverges (has no finite
limit).
% Emacs 25.0.50.1 (Org mode 8.2.2)
\end{document}