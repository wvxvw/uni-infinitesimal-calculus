% Created 2015-04-24 Fri 17:33
\documentclass[11pt]{article}
\usepackage[utf8]{inputenc}
\usepackage[T1]{fontenc}
\usepackage{fixltx2e}
\usepackage{graphicx}
\usepackage{longtable}
\usepackage{float}
\usepackage{wrapfig}
\usepackage{rotating}
\usepackage[normalem]{ulem}
\usepackage{amsmath}
\usepackage{textcomp}
\usepackage{marvosym}
\usepackage{wasysym}
\usepackage{amssymb}
\usepackage{hyperref}
\tolerance=1000
\usepackage[utf8]{inputenc}
\usepackage[usenames,dvipsnames]{color}
\usepackage[backend=bibtex, style=numeric]{biblatex}
\usepackage{commath}
\usepackage{tikz}
\usetikzlibrary{shapes,backgrounds}
\usepackage{marginnote}
\usepackage{listings}
\usepackage{color}
\usepackage{enumerate}
\hypersetup{urlcolor=blue}
\hypersetup{colorlinks,urlcolor=blue}
\addbibresource{bibliography.bib}
\setlength{\parskip}{16pt plus 2pt minus 2pt}
\definecolor{codebg}{rgb}{0.96,0.99,0.8}
\definecolor{codestr}{rgb}{0.46,0.09,0.2}
\author{Oleg Sivokon}
\date{\textit{<2015-04-03 Fri>}}
\title{Assignment 13, Infinitesimal Calculus}
\hypersetup{
  pdfkeywords={Infinitesimal Calculus, Assignment, Definition of Limits},
  pdfsubject={Third asssignment in the course Infinitesimal Calculus},
  pdfcreator={Emacs 25.0.50.1 (Org mode 8.2.2)}}
\begin{document}

\maketitle
\tableofcontents


\lstset{ %
  backgroundcolor=\color{codebg},
  basicstyle=\ttfamily\scriptsize,
  breakatwhitespace=false,         % sets if automatic breaks should only happen at whitespace
  breaklines=false,
  captionpos=b,                    % sets the caption-position to bottom
  commentstyle=\color{mygreen},    % comment style
  framexleftmargin=10pt,
  xleftmargin=10pt,
  framerule=0pt,
  frame=tb,                        % adds a frame around the code
  keepspaces=true,                 % keeps spaces in text, useful for keeping indentation of code (possibly needs columns=flexible)
  keywordstyle=\color{blue},       % keyword style
  showspaces=false,                % show spaces everywhere adding particular underscores; it overrides 'showstringspaces'
  showstringspaces=false,          % underline spaces within strings only
  showtabs=false,                  % show tabs within strings adding particular underscores
  stringstyle=\color{codestr},     % string literal style
  tabsize=2,                       % sets default tabsize to 2 spaces
}

\clearpage

\section{Problems}
\label{sec-1}

\subsection{Problem 1}
\label{sec-1-1}
Given the sequence $(a_n)$ defined as $0 < a_1 < 6$ and $a_{n+1} = \sqrt{6a_n}$
for all $n$.
\begin{enumerate}
\item Prove that the sequence converges.
\item Find $\lim_{n \to \infty} a_n$.
\end{enumerate}

\subsubsection{Answer 1}
\label{sec-1-1-1}
First, observe that every next element of the sequence $(a_n)$ is bigger than
the previous.  Using mathematical induction we can show that:

\textbf{Base step}: $\sqrt{a_1} < \sqrt{a_2}$ because $a_1 < 6$ and thus
$\sqrt{a_1} < \sqrt{6}$, consequently (from order preservation under multiplication)
$\sqrt{a_1} \times \sqrt{a_1} < \sqrt{6} \times \sqrt{a_1}$ and hence
$\sqrt{a_1} < \sqrt{a_2}$.

\textbf{Inductive step}:
\begin{equation*}
  \begin{aligned}
    a_k                        &< \sqrt{6 \times a_{k+1}} \\
                               &\textit{Using induction hypothesis, $k = n - 1$} \\
    \sqrt{6 \times a_k}        &< \sqrt{6 \times a_{k+1}} \\
    \sqrt{6} \times \sqrt{a_k} &< \sqrt{6} \times \sqrt{a_{k+1}} \\
                               &\textit{Order preservation under multiplication} \\
    \sqrt{a_k}                 &< \sqrt{a_{k+1}} \\
    a_k                 &< a_{k+1}
  \end{aligned}
\end{equation*}

Hence, by mathematical induction, the sequence $(a_n)$ is an increasing one.

Now we demonstrate that this sequence is bounded above and it has a supremum.
Again, using induction, we can choose 6 to be the upper bound.

\textbf{Base step}: $a_1 < 6$ by definition.

\textbf{Inductive step}:
\begin{equation*}
  \begin{aligned}
    a_k                 &< 6 \\
                        &\textit{Using induction hypothesis, $k = n - 1$} \\
    \sqrt{6 \times a_k} &< \sqrt{6 \times 6} \\
    \sqrt{6} \times \sqrt{a_k} &< \sqrt{6} \times \sqrt{6} \\
                        &\textit{Order preservation under multiplication} \\
    \sqrt{a_k}          &< \sqrt{6} \\
    a_k &< 6
  \end{aligned}
\end{equation*}

Hence, by mathematical induction, $(a_n)$ has an upper bound. It is trivial
to see that this is also the supremum, since $a_1$ is defined to be strictly
less than 6, so no smaller upper bound can exist.

Having showed both these properties, we can now use the monotone sequence
theorem to claim that this sequence converges.  Accidentally, we also found
the limit of this sequence, which wasn't due until the next question.
\subsubsection{Answer 2}
\label{sec-1-1-2}
We have already found the limit of $(a_n)$ as $n$ approaches infinity.  This
is given by a corollary from the monotone sequence theorem: the supremum
is also the limit of a monotone sequence.  Thus the answer is 6.
\subsection{Problem 2}
\label{sec-1-2}
Find the limits of:

\begin{enumerate}
\item \begin{equation*}
  \lim_{n \to \infty} \frac{(-5)^n - 2^n + 2}{3^n + (-2)^n - 2}.
\end{equation*}

\item \begin{equation*}
  \lim_{n \to \infty} \frac{3^n + (-2)^n - 2}{(-5)^n - 2^n + 2}.
\end{equation*}

\item \begin{equation*}
  \lim_{n \to \infty} (\lfloor 2n \rfloor - 2 \lfloor n \rfloor).
\end{equation*}

\item \begin{equation*}
  \lim_{n \to \infty} \frac{\sqrt[n]{n!}}{n}.
\end{equation*}
\end{enumerate}

\subsubsection{Answer 3}
\label{sec-1-2-1}
We can see that the denominator of this limit will be a positive
integer for $n > 1$.  The numerator will alternate between positive
and negative values, where the difference will only increase with
time since $(-5)^2n > 2^n$ and $(-5)^{2n-1} > 2^n$, thus this expression
has two limit points: $\infty$ and $-\infty$.
\subsubsection{Answer 4}
\label{sec-1-2-2}
The same reasoning we put forward in \ref{sec-1-2-1} applies here, but now the
numerator will be some positive integer, smaller than the denominator.
Thus this expression will have a single limit at 0.
\subsubsection{Answer 5}
\label{sec-1-2-3}
Here I will assume $n$ to be real, otherwise the limit is trivially 0.
In this case we can split the sequence into the following subsequences:
\begin{equation*}
  \begin{aligned}
    (a_n) = \{\lfloor x \rfloor | x \in \mathbb{R}: x \pmod 2 < 0.5 \} \\
    (b_n) = \{\lfloor x \rfloor | x \in \mathbb{R}: x \pmod 2 \geq 0.5 \}
  \end{aligned}
\end{equation*}

It's easy to see that $(a_n) \cup (b_n)$ gives the initial set.  These
subsets have constant limits $\{0, 1\}$ as is given by their generating
formulae: $\lfloor x \pmod 2 < 0.5 \rfloor$ and $\lfloor x \pmod 2 < 0.5
    \rfloor$ resp.  But the whole expression doesn't have a single limit point,
since we can always find as many points as we like that would not be an
$\epsilon$ distance from one of the limit points found.
\subsubsection{Answer 6}
\label{sec-1-2-4}
We can show this expression approaches 1 as $n$ gets larger using the
following algebraic transformations:
\begin{equation*}
  \begin{aligned}
    \frac{\sqrt[n]{n!}}{n} &= \sqrt[n]{\frac{n!}{n^n}} \\
                           &= \sqrt[n]{\prod_{k=1}^n\frac{n - k}{n}}
  \end{aligned}
\end{equation*}

Since $n - k < n$, $\frac{n - k}{n} < 1$.  The product of arbitrary many
terms all of which are less than one will be less than one.  And will
become smaller as the number of terms grows.  However, the nth root
will become progressively large as the fractional term becomes smaller,
yet never exceeding 1.  Thus the limit of this expression is 1.
\subsection{Problem 3}
\label{sec-1-3}
Let $(a_n)$ and $(b_n)$ be sequences both bounded from above.
\begin{enumerate}
\item Prove $\sup \{a_n + b_n | n \in \mathbb{N}\} \leq \sup
      \{a_n | n \in \mathbb{N}\} + \sup \{b_n | n \in \mathbb{N}\}$.
\item Find such $(a_n)$ and $(b_n)$ which are equal under the condition
given in (1).
\item Find such $(a_n)$ and $(b_n)$ for which the inequality given in
(1) holds.
\end{enumerate}

\subsubsection{Answer 7}
\label{sec-1-3-1}

\subsubsection{Answer 8}
\label{sec-1-3-2}
An example of $(a_n) = \frac{1}{n}$ whose supremum is 1 and $(b_n) =
    \frac{2}{n}$.  Since they both reach their maximum output at $n=1$, the sum
of their suprema and the supremum of their sum is the same.
\subsubsection{Answer 9}
\label{sec-1-3-3}
To contrast \ref{sec-1-3-2}, $(a_n) = \frac{1}{n}$ and $(b_n) = \frac{1}{n - 1}$
reach their suprema at different times, ($(b_n)$ isn't even defined at the
time $(a_n)$ reaches its maximum), the supremum of their sum is thus 1.5,
but the sum of their suprema is 2.
\subsection{Problem 4}
\label{sec-1-4}
Let $(a_n) = n - \lfloor \sqrt{n} \rfloor^2$.
\begin{enumerate}
\item Prove that $(a_n)$ is bounded from below.
\item Prove that 0 is a limit point of $(a_n)$.
\item Find $\inf \{a_n | n \in \mathbb{N}\}$, $\liminf_{n \to \infty} a_n$.
Establish whether $(a_n)$ has minima.
\item Given natural number $\ell$, prove that almost for all $n$ it holds
that $n < \sqrt{n^2 + \ell} < n + 1$.
\item Prove that every natural number is a limit point of $(a_n)$.
\item Is $(a_n)$ bounded from above?
\item Find $\limsup_{n \to \infty} a_n$.
\end{enumerate}

\subsubsection{Answer 10}
\label{sec-1-4-1}

\subsubsection{Answer 11}
\label{sec-1-4-2}

\subsubsection{Answer 12}
\label{sec-1-4-3}
\subsection{Problem 5}
\label{sec-1-5}
Let $(a_n)$ and $(b_n)$ be sequences.
\begin{enumerate}
\item Assume $\lim_{n \to \infty}(a_n + b_n) = L$ for some finite $L$.
Prove that if $(a_n)$ is bounded then $(b_n)$ is bounded too.
\item Assume $\lim_{n \to \infty}(a_n + b_n) = L$ for some finite $L$.
Prove that if $(a_n)$ has a limit point $a$, then $L - a$ is a
limit point of $(b_n)$.
\item Assume $(a_n)$ has 20106 limit points and $(b_n)$ has 20474 limit
points. Prove that $(a_n + b_n)$ diverges.
\end{enumerate}
% Emacs 25.0.50.1 (Org mode 8.2.2)
\end{document}