% Created 2015-03-17 Tue 11:34
\documentclass[11pt]{article}
\usepackage[utf8]{inputenc}
\usepackage[T1]{fontenc}
\usepackage{fixltx2e}
\usepackage{graphicx}
\usepackage{longtable}
\usepackage{float}
\usepackage{wrapfig}
\usepackage{rotating}
\usepackage[normalem]{ulem}
\usepackage{amsmath}
\usepackage{textcomp}
\usepackage{marvosym}
\usepackage{wasysym}
\usepackage{amssymb}
\usepackage{hyperref}
\tolerance=1000
\usepackage[utf8]{inputenc}
\usepackage[usenames,dvipsnames]{color}
\usepackage[backend=bibtex, style=numeric]{biblatex}
\usepackage{commath}
\usepackage{tikz}
\usetikzlibrary{shapes,backgrounds}
\usepackage{marginnote}
\usepackage{listings}
\usepackage{color}
\usepackage{enumerate}
\hypersetup{urlcolor=blue}
\hypersetup{colorlinks,urlcolor=blue}
\addbibresource{bibliography.bib}
\setlength{\parskip}{16pt plus 2pt minus 2pt}
\definecolor{codebg}{rgb}{0.96,0.99,0.8}
\definecolor{codestr}{rgb}{0.46,0.09,0.2}
\author{Oleg Sivokon}
\date{\textit{<2015-03-14 Sat>}}
\title{Assignment 11, Infinitesimal Calculus}
\hypersetup{
  pdfkeywords={Infinitesimal Calculus, Assignment, Properties of Numbers},
  pdfsubject={First asssignment in the course Infinitesimal Calculus},
  pdfcreator={Emacs 25.0.50.1 (Org mode 8.2.2)}}
\begin{document}

\maketitle
\tableofcontents


\lstset{ %
  backgroundcolor=\color{codebg},
  basicstyle=\ttfamily\scriptsize,
  breakatwhitespace=false,         % sets if automatic breaks should only happen at whitespace
  breaklines=false,
  captionpos=b,                    % sets the caption-position to bottom
  commentstyle=\color{mygreen},    % comment style
  framexleftmargin=10pt,
  xleftmargin=10pt,
  framerule=0pt,
  frame=tb,                        % adds a frame around the code
  keepspaces=true,                 % keeps spaces in text, useful for keeping indentation of code (possibly needs columns=flexible)
  keywordstyle=\color{blue},       % keyword style
  showspaces=false,                % show spaces everywhere adding particular underscores; it overrides 'showstringspaces'
  showstringspaces=false,          % underline spaces within strings only
  showtabs=false,                  % show tabs within strings adding particular underscores
  stringstyle=\color{codestr},     % string literal style
  tabsize=2,                       % sets default tabsize to 2 spaces
}

\clearpage

\section{Problems}
\label{sec-1}

\subsection{Problem 1}
\label{sec-1-1}
\begin{enumerate}
\item Prove that for any natural number $n$ it holds that:
\begin{equation*}
  4^n \geq {2n \choose n}.
\end{equation*}

\item Prove by induction, or in any other way, that for any natural number $n$
it holds that:
\begin{equation*}
   {2n \choose n} \geq \frac{4^n}{2n+1}.
\end{equation*}
\end{enumerate}

\subsubsection{Answer 1}
\label{sec-1-1-1}
\textbf{Discussion:} The idea behind the proof is to show that ${2n \choose n}$ is
a term in the binomial expansion representing $4^n$.  Since all other terms
in this expansion will be non-negative, then $4^n$ will be at least as big
as ${2n \choose n}$.

\textbf{Proof:} Put $4^n=(1+1)^{2n}$, using binomial formula obtains:
\begin{align*}
  (1+1)^{2n} &= \sum_{k=0}^{2n} {2n \choose k}1^{2n}1^{2n-k} \\
            &= {2n \choose n} + \sum_{k=0, k\neq n}^{2n} {2n \choose k}.
\end{align*}

We know that ${2n \choose n}$ is a term of binomial expansion because we
know that $k_i \leq 2n$, which implies that since $k_i, n$ are natural numbers,
there exists $k_i = n$.  Besides, there might exist other terms in binomial
expansion which are guaranteed to be non-negative.  Hence,
$4^n \geq {2n \choose n}$.
\subsubsection{Answer 2}
\label{sec-1-1-2}
\textbf{Discussion:} In order to make the proof a little less verbose, I will prove
a stronger claim, viz. ${2n \choose n} \geq \frac{4^n}{2n}$.  Since $n$ is
positive, $2n < 2n +1$, hence $\frac{4^n}{2n} > \frac{4^n}{2n + 1}$.  The
proof will proceed by induction on $n$.  First I will find a factor s.t.
multiplying it with $S_{n-1}$ I will obtain $S_n$, and then multiply it
with the $\frac{4^n}{2n}$ to show that it will necessary be at leas as large
as $\frac{4^{n+1}}{2(n+1)}$.

\textbf{Proof:} Using mathematical induction, let's first prove the base step,
where $n=1$:
\begin{align*}
  { 2*1 \choose 1}    &\geq \frac{4^1}{2*1} \iff \\
  \frac{2!}{1!(2-1)!} &\geq \frac{4}{2} \iff \\
  \frac{2}{1}         &\geq 2 \iff \\
  2                   &\geq 2.
\end{align*}

Now, to the inductive step (for $n>1$), some useful simplification first:
\begin{equation}
  \begin{aligned}
    {2n \choose n} &= \frac{(2n)!}{n!(2n - n)!} \\
    &= \frac{(2n!)}{n!n!}. \\
  \end{aligned}
  \label{eq:base}
\end{equation}

Invoking inductive hypothesis ${2(n+1) \choose (n+1)} \geq \frac{4^n}{2n}$:
\begin{equation}
  \begin{aligned}
    {2(n+1) \choose n+1} &= \frac{(2(n+1))!}{(n+1)!(2(n+1) - n+1)!} \\
    &= \frac{(2n!)(2n+1)(2n+2)}{n!(n+1)n!(n+1)} \\
    &= \frac{(2n!)(2n+1)2(n+1)}{n!(n+1)n!(n+1)} \\
    &= \frac{(2n!)(2n+1)2}{n!n!(n+1)}.
  \end{aligned}
  \label{eq:step}
\end{equation}

Dividing $\ref{eq:step}$ by $\ref{eq:base}$ gives us the factor $\frac{(2n+1)2}{(n+1)}$. Thus:
\begin{align*}
  \frac{(2n+1)2}{n+1} \times \frac{4^n}{2n} &\geq \frac{4^{n+1}}{2(n+1)} \\
  \frac{(2n+1)2*4^n}{2n(n+1)} &\geq \frac{4^{n+1}}{2(n+1)} \\
  \frac{(2n+1)2*4^n}{n} &\geq 4^{n+1} \\
  \frac{(2n+1)2*4^n}{n} &\geq 4^{n}*4 \\
  \frac{(2n+1)2}{n} &\geq 4 \\
  \frac{4n+2}{n} &\geq 4 \\
  4 + \frac{2}{n} &\geq 4 \\
\end{align*}

Since $n > 1$, $\frac{2}{n}$ is positive, hence the inequality holds.
This completes the inductive step.  Hence, by using mathematical induction
the proof is complete.
\subsection{Problem 2}
\label{sec-1-2}
\begin{enumerate}
\item Given $k, l \in \mathbb{N}$, prove that $a=k+l\sqrt{2}$ is irrational.
\item Prove that for every natural number $n$ it holds that:
\begin{equation*}
   \sum_{i=0}^n \sqrt{2}^i
\end{equation*}

is irrational.
\end{enumerate}

\subsubsection{Answer 3}
\label{sec-1-2-1}
\textbf{Discussion:} One way to see that summation of rational with irrational
cannot produce a rational number is through invoking field axioms: summation
must send the sum to the field of rationals, which would imply that the
inverses of summands must be rationals too.  This would also require summands
to be rationals, but that's not possible.

\textbf{Proof:} Suppose, for contradiction that $k+l\sqrt{2}=a$ is rational, then
\begin{equation*}
  \begin{aligned}
    k+l\sqrt{2} &= a & \textrm{Given} \\
    k+(-k)+l\sqrt{2} &= a-k &
    \textrm{$k$ must have additive inverse in $\mathbb{Q}$} \\
    0+l\sqrt{2} &= a-k \\
    l^{-1}l\sqrt{2} &= l^{-1}(a-k) &
    \textrm{$l$ must have multiplicative inverse in $\mathbb{Q}$} \\
    1\sqrt{2} &= l^{-1}(a-k) \\
    \sqrt{2} &= l^{-1}(a-k).
  \end{aligned}
\end{equation*}


$l^{-1}, q$ and $k$ are all rationals, rationals are closed under multiplication
and addition, hence $l^{-1}(q-k)$ must be rational, but $\sqrt{2}$ is not.
Contradiction.  Hence $a=k+l\sqrt{2}$ is irrational.
\subsubsection{Answer 4}
\label{sec-1-2-2}
\textbf{Discussion:} The way to see that this statement is true is to divide the sequence
into odd and even terms.  All even terms will produce rationals (even poverse of
square root of two will be rational).  While all odd terms will produce irrational
numbers (a product of even number of square roots of two will give a rational, but
them multiplied with an irrational number will give an irrational).  Since $n$
must be at least one (and thus we are guaranteed to have at least one odd term in
this sequence), the sum of the sequence will always be irrational.

\textbf{Proof:} Let's rewrite this sum as two sums of the form:
\begin{equation*}
  \sum_{i=0}^{\lfloor n/2 \rfloor} \sqrt{2}^{2i} +
  \sum_{i=0}^{\lfloor (n+1)/2 \rfloor} \sqrt{2}^{2i+1}.
\end{equation*}

It is easy to see that the first term is a sum of powers of 2, viz:
$\sqrt{2}^2+\sqrt{2}^4+\hdots+\sqrt{2}^{\lfloor n/2 \rfloor}$, which is just
$2+4+\hdots+2^{\lfloor n/2-1 \rfloor}$.  Similarly, the terms of the other
sum can be expressed as $\sqrt{2}^1+\sqrt{2}^3+\hdots+\sqrt{2}^{\lfloor
    (n+1)/2 \rfloor}$.

Let's give names to the sequences we outlined: $S_1=\sum_{i=0}^{\lfloor n/2
    \rfloor} \sqrt{2}^{2i}$ and $S_2=\sum_{i=0}^{\lfloor (n+1)/2 \rfloor}
    \sqrt{2}^{2i+1}$.  Now, for contradiction, assume $S_2$ to be rational.
Then let's define $\div$ to be the element-wise division of sequences,
i.e. for sequences $A$ and $B$ of length $n$, $\div$ is defined to be:
\begin{equation*}
  \sum_{i=0}^n \frac{A_i}{B_i}.
\end{equation*}

It's easy to see that if all elements of $A$ and $B$ are rational, then $A
    \div B$ is rational too (because we only used addition and multiplication,
which are known to be closed over rationals).

Observe that $S_2 \div S_3=\sum_{i=0}^{\lfloor (n+1)/2 \rfloor} \sqrt{2}$,
i.e. a sum of $n/2$ square roots of 2, which is the same as $n/2*\sqrt{2}$,
but we've just showed that the product of a rational and irrational cannot be
rational (in previous question). Hence $S_2$ must be irrational, contrary to
assumed.  Hence it must be that $\sum_{i=0}^n \sqrt{2}^i$ is irrational.
This completes the proof.
\subsection{Problem 3}
\label{sec-1-3}
\begin{enumerate}
\item Given real numbers $a$ and $b$ prove that if
\begin{equation*}
   \frac{\abs{a}}{2}>\abs{b-\frac{a}{2}},
\end{equation*}

then
\begin{equation*}
   \abs{b-a}<\abs{a}.
\end{equation*}
\end{enumerate}

\subsubsection{Answer 5}
\label{sec-1-3-1}
\textbf{Discussion:} The proof will be based on invariance of order under multiplication,
in particular, it will rely on the fact that $\abs{x} < \abs{y} \iff x^2 < y^2$.
This will allow us to solve the inequality without splitting it into several
cases.

\textbf{Proof:} First, let's simplify the expression:
\begin{equation*}
  \begin{aligned}
    \frac{\abs{a}}{2} &> \abs{b-\frac{a}{2}} 
    & \textrm{Given} \\
    \abs{a}           &> 2\abs{b-\frac{a}{2}}
    & \textrm{Invariance of order under multiplication} \\
    \abs{a}           &> \abs{2b-a}
    & \textrm{Distributivity of multiplication over addition} \\
    \abs{2b-a}        &< \abs{a} \\
    \abs{2b-a}^2      &< \abs{a}^2
    & \textrm{Invariance under exponentiation} \\
    (2b-a)^2          &< a^2
    & \textrm{Squares are always positive} \\
    4b^2 - 4ba + a^2  &< a^2 \\
    4b^2 - 4ba        &< 0 \\
    4b^2              &< 4ba \\
    b^2               &< ba
    & \textrm{Invariance under multiplication}
  \end{aligned}
\end{equation*}

Next, let's perform similar operations on $\abs{b-a}<\abs{a}$
\begin{equation*}
  \begin{aligned}
    \abs{b - a}     &< \abs{a}
    & \textrm{Given} \\
    \abs{b - a}^2   &< \abs{a}^2
    & \textrm{Invariance under exponentiation} \\
    (b - a)^2       &< a^2
    & \textrm{Squares are always positive} \\
    b^2 - 2ba + a^2 &< a^2 \\
    b^2 - 2ba       &< 0 \\
    b^2             &< 2ba.
    \end{aligned}
\end{equation*}


From transitivity of order it follows that $b^2 < ba < 2ba$, i.e.
$b^2 < 2ba$, but this is exactly the condition we set out to prove in the
very beginning.  Thus, the proof is complete.
\subsection{Problem 4}
\label{sec-1-4}
Given $a, b, c \in \mathbb{R}$,
\begin{enumerate}
\item Prove that if $a > 0$ and $a + b > a + c$, then $b > c$.
\item Prove that if $a > 0$ and $ab > ac$, then $b > c$.
\item Prove that if $\abs{a} > \abs{b}$ iff $a^2 > b^2$.
\item Prove that if $b > c$ and $\abs{a-b} > \abs{a-c}$, then $b > a$.
\item Show (my means of example) that from $b > c$ and $b > a$ it doesn't
follow that $\abs{a-b} > \abs{a-c}$.
\end{enumerate}

\subsubsection{Answer 6}
\label{sec-1-4-1}
\subsubsection{Answer 7}
\label{sec-1-4-2}
\subsubsection{Answer 8}
\label{sec-1-4-3}
\subsubsection{Answer 9}
\label{sec-1-4-4}
\subsubsection{Answer 10}
\label{sec-1-4-5}
\subsection{Problem 5}
\label{sec-1-5}
Solve the equation:
\begin{equation*}
   \lfloor \abs{x+1}-\abs{x-1} \rfloor = x.
\end{equation*}


\subsubsection{Answer 11}
\label{sec-1-5-1}
\subsection{Problem 6}
\label{sec-1-6}
\textbf{Definition:} set $A$ of real numbers is called \textbf{dense in interval} $I$ if
for every $x, y \in I$ s.t. $x < y$ there exists $a \in A$ such that
$x < a < y$.

\begin{enumerate}
\item Let $A$ be dense in interval $[0,1]$, prove that set 
      $B=\{na|a \in A, n \in \mathbb{N}\}$ is dense in interval $[0, \infty)$.
\item Let $A=\mathbb{R}$, prove that $A$ isn't dense in $I$ iff exists an
open interval $(x, y)$ in $I$, such that $A \cap (x, y) = \emptyset$.
\item Let $A$ be the real numbers in interval $[0,1]$, prove that the set
      $C=\{\frac{a+1}{n^2} | a \in A, n \in \mathbb{N}\}$ isn't dense in
      $[0,1]$.
\end{enumerate}

\subsubsection{Answer 12}
\label{sec-1-6-1}
\subsubsection{Answer 13}
\label{sec-1-6-2}
\subsubsection{Answer 14}
\label{sec-1-6-3}
% Emacs 25.0.50.1 (Org mode 8.2.2)
\end{document}