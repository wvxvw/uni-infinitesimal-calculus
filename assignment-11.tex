% Created 2015-03-14 Sat 12:10
\documentclass[11pt]{article}
\usepackage[utf8]{inputenc}
\usepackage[T1]{fontenc}
\usepackage{fixltx2e}
\usepackage{graphicx}
\usepackage{longtable}
\usepackage{float}
\usepackage{wrapfig}
\usepackage{rotating}
\usepackage[normalem]{ulem}
\usepackage{amsmath}
\usepackage{textcomp}
\usepackage{marvosym}
\usepackage{wasysym}
\usepackage{amssymb}
\usepackage{hyperref}
\tolerance=1000
\usepackage[utf8]{inputenc}
\usepackage[usenames,dvipsnames]{color}
\usepackage[backend=bibtex, style=numeric]{biblatex}
\usepackage{commath}
\usepackage{tikz}
\usetikzlibrary{shapes,backgrounds}
\usepackage{marginnote}
\usepackage{listings}
\usepackage{color}
\usepackage{enumerate}
\hypersetup{urlcolor=blue}
\hypersetup{colorlinks,urlcolor=blue}
\addbibresource{bibliography.bib}
\setlength{\parskip}{16pt plus 2pt minus 2pt}
\definecolor{codebg}{rgb}{0.96,0.99,0.8}
\definecolor{codestr}{rgb}{0.46,0.09,0.2}
\author{Oleg Sivokon}
\date{\textit{<2015-03-14 Sat>}}
\title{Assignment 11, Infinitesimal Calculus}
\hypersetup{
  pdfkeywords={Infinitesimal Calculus, Assignment, Properties of Numbers},
  pdfsubject={First asssignment in the course Infinitesimal Calculus},
  pdfcreator={Emacs 25.0.50.1 (Org mode 8.2.2)}}
\begin{document}

\maketitle
\tableofcontents


\lstset{ %
  backgroundcolor=\color{codebg},
  basicstyle=\ttfamily\scriptsize,
  breakatwhitespace=false,         % sets if automatic breaks should only happen at whitespace
  breaklines=false,
  captionpos=b,                    % sets the caption-position to bottom
  commentstyle=\color{mygreen},    % comment style
  framexleftmargin=10pt,
  xleftmargin=10pt,
  framerule=0pt,
  frame=tb,                        % adds a frame around the code
  keepspaces=true,                 % keeps spaces in text, useful for keeping indentation of code (possibly needs columns=flexible)
  keywordstyle=\color{blue},       % keyword style
  showspaces=false,                % show spaces everywhere adding particular underscores; it overrides 'showstringspaces'
  showstringspaces=false,          % underline spaces within strings only
  showtabs=false,                  % show tabs within strings adding particular underscores
  stringstyle=\color{codestr},     % string literal style
  tabsize=2,                       % sets default tabsize to 2 spaces
}

\clearpage

\section{Problems}
\label{sec-1}

\subsection{Problem 1}
\label{sec-1-1}
\begin{enumerate}
\item Prove that for any natural number $n$ it holds that:
\begin{equation*}
  4^n \geq {2n \choose n}.
\end{equation*}

\item Prove by induction, or in any other way, that for any natural number $n$
it holds that:
\begin{equation*}
   {2n \choose n} \geq \frac{4^n}{2n+1}.
\end{equation*}
\end{enumerate}

\subsubsection{Answer 1}
\label{sec-1-1-1}
\subsubsection{Answer 2}
\label{sec-1-1-2}
\subsection{Problem 2}
\label{sec-1-2}
\begin{enumerate}
\item Given $k, l \in \mathbb{N}$, prove that $a=k+l\sqrt{2}$ is irrational.
\item Prove that for every natural number $n$ it holds that:
\begin{equation*}
   \sum_{i=0}^n \sqrt{2}^i
\end{equation*}

is irrational.
\end{enumerate}

\subsubsection{Answer 3}
\label{sec-1-2-1}
\subsubsection{Answer 4}
\label{sec-1-2-2}
\subsection{Problem 3}
\label{sec-1-3}
\begin{enumerate}
\item Given real numbers $a$ and $b$ prove that if
\begin{equation*}
   \frac{\abs{a}}{2}>\abs{b-\frac{a}{2}},
\end{equation*}

then
\begin{equation*}
   \abs{b-a}<\abs{a}.
\end{equation*}
\end{enumerate}

\subsubsection{Answer 5}
\label{sec-1-3-1}

\subsection{Problem 4}
\label{sec-1-4}
Given $a, b, c \in \mathbb{R}$,
\begin{enumerate}
\item Prove that if $a > 0$ and $a + b > a + c$, then $b > c$.
\item Prove that if $a > 0$ and $ab > ac$, then $b > c$.
\item Prove that if $\abs{a} > \abs{b}$ iff $a^2 > b^2$.
\item Prove that if $b > c$ and $\abs{a-b} > \abs{a-c}$, then $b > a$.
\item Show (my means of example) that from $b > c$ and $b > a$ it doesn't
follow that $\abs{a-b} > \abs{a-c}$.
\end{enumerate}

\subsubsection{Answer 6}
\label{sec-1-4-1}
\subsubsection{Answer 7}
\label{sec-1-4-2}
\subsubsection{Answer 8}
\label{sec-1-4-3}
\subsubsection{Answer 9}
\label{sec-1-4-4}
\subsubsection{Answer 10}
\label{sec-1-4-5}
\subsection{Problem 5}
\label{sec-1-5}
Solve the equation:
\begin{equation*}
   \lfloor \abs{x+1}-\abs{x-1} \rfloor = x.
\end{equation*}


\subsubsection{Answer 11}
\label{sec-1-5-1}
\subsection{Problem 6}
\label{sec-1-6}
\textbf{Definition:} set $A$ of real numbers is called \textbf{dense in interval} $I$ if
for every $x, y \in I$ s.t. $x < y$ there exists $a \in A$ such that
$x < a < y$.

\begin{enumerate}
\item Let $A$ be dense in interval $[0,1]$, prove that set 
      $B=\{na|a \in A, n \in \mathbb{N}\}$ is dense in interval $[0, \infty)$.
\item Let $A=\mathbb{R}$, prove that $A$ isn't dense in $I$ iff exists an
open interval $(x, y)$ in $I$, such that $A \cap (x, y) = \emptyset$.
\item Let $A$ be the real numbers in interval $[0,1]$, prove that the set
      $C=\{\frac{a+1}{n^2} | a \in A, n \in \mathbb{N}\}$ isn't dense in
      $[0,1]$.
\end{enumerate}

\subsubsection{Answer 12}
\label{sec-1-6-1}
\subsubsection{Answer 13}
\label{sec-1-6-2}
\subsubsection{Answer 14}
\label{sec-1-6-3}
% Emacs 25.0.50.1 (Org mode 8.2.2)
\end{document}