% Created 2015-06-27 Sat 19:27
\documentclass[a4paper]{article}
\usepackage[utf8]{inputenc}
\usepackage[T1]{fontenc}
\usepackage{fixltx2e}
\usepackage{graphicx}
\usepackage{longtable}
\usepackage{float}
\usepackage{wrapfig}
\usepackage{rotating}
\usepackage[normalem]{ulem}
\usepackage{amsmath}
\usepackage{textcomp}
\usepackage{marvosym}
\usepackage{wasysym}
\usepackage{amssymb}
\usepackage{capt-of}
\usepackage{hyperref}
\tolerance=1000
\usepackage[utf8]{inputenc}
\usepackage[usenames,dvipsnames]{color}
\usepackage{commath}
\usepackage{tikz}
\usetikzlibrary{shapes,backgrounds}
\usepackage{marginnote}
\usepackage{listings}
\usepackage{color}
\usepackage{enumerate}
\hypersetup{urlcolor=blue}
\hypersetup{colorlinks,urlcolor=blue}
\setlength{\parskip}{16pt plus 2pt minus 2pt}
\definecolor{codebg}{rgb}{0.96,0.99,0.8}
\definecolor{codestr}{rgb}{0.46,0.09,0.2}
\DeclareMathOperator{\Dom}{Dom}
\allowdisplaybreaks[4]
\author{Oleg Sivokon}
\date{\textit{<2015-06-24 Wed>}}
\title{Assignment 18, Infinitesimal Calculus}
\hypersetup{
 pdfauthor={Oleg Sivokon},
 pdftitle={Assignment 18, Infinitesimal Calculus},
 pdfkeywords={Infinitesimal Calculus, Assignment, Limits of functions},
 pdfsubject={Fourth asssignment in the course Infinitesimal Calculus},
 pdfcreator={Emacs 25.0.50.1 (Org mode 8.3beta)}, 
 pdflang={English}}
\begin{document}

\maketitle
\tableofcontents

\definecolor{codebg}{rgb}{0.96,0.99,0.8}
\lstnewenvironment{maxima}{%
  \lstset{backgroundcolor=\color{codebg},
    escapeinside={(*@}{@*)},
    aboveskip=20pt,
    showstringspaces=false,
    frame=single,
    framerule=0pt,
    basicstyle=\ttfamily\scriptsize,
    columns=fixed}}{}
}
\makeatletter
\newcommand{\verbatimfont}[1]{\renewcommand{\verbatim@font}{\ttfamily#1}}
\makeatother
\verbatimfont{\small}%
\clearpage

\section{Problems}
\label{sec:orgheadline15}

\subsection{Problem 1}
\label{sec:orgheadline2}
Given \(P(x) = x^4 + a_4x^3 + a_3x^2 + a_2x + a_0\) is a polynomial with one
non-zero real root, prove that it has at least one more real root.

\subsubsection{Answer 1}
\label{sec:orgheadline1}
Since we are given \(P(x)\) has a root, let's call it \(r\), then from factor
theorem we also know that there exists \((x - r)D(x) = P(x)\) where \(D(x)\) is
a third degree polynomial with a non-zero leading coefficient.  According to
fundamental theorem of calculus, each root must have a conjugate pair.
Since third degree polynomial has three roots, then by pigeonhole principle,
one of these roots is equal to its conjugate, which is why this root must be
real.  Hence the original quadratic polynomial has at least one more real
root (and four roots all in all).

\subsection{Problem 2}
\label{sec:orgheadline4}
Let \(f\) be a function differentiable at \((a, b)\).  Let \(x_0 \in (a, b)\).
Prove that there exists a sequence \((x_n)\) s.t. \(x_n \neq x_0\) for all \(n\),
\(\lim_{n \to \infty}x_n = x_0\), \(\lim_{n \to \infty}f'(x_n) = f'(x_0)\).

\subsubsection{Answer 2}
\label{sec:orgheadline3}
Let's pick \(x_0\).  From continuity of reals, we have that no matter where in
\((a, b)\) \(x_0\) is, we can always find two intervlas \((a, x_0)\) and \((x_0,
    b)\).  Let's pick a suitable derivative function.  For example, let \(f'(x) =
    f'(x_0)\), i.e. a constant function.  Now we can construct the required
sequence.  \((x_n) = x_0 - \frac{x_0 - a}{n + 2}\) meets our requirements.
Since \(\lim_{n \to \infty}x_0 = x_0\) and \(\lim_{n \to \infty}\frac{x_0 -
    a}{n + 2} = 0\).

\subsection{Problem 3}
\label{sec:orgheadline6}
Let \(f\) be differenetiable at \([0, \frac{\pi}{2}]\) s.t. \(0 \leq f(x) \leq 1\) for
all \(x\) in this interval.  Prove that there exists a point \(x\) in this interval
such that \(f'(x) = \sin x\).

\subsubsection{Answer 3}
\label{sec:orgheadline5}
From Lagrange mean value theorem we are guaranteed a point \(c\) s.t.
\begin{align*}
  f'(c) = \frac{f(a) - f(b)}{a - b}
\end{align*}

Put \(a = 0\), \(b = \frac{\pi}{2}\).  Then it follows that \(0 < f'(c) < 1\)
since \(f(a) \geq a\) and \(f(b) \leq b\).  Since sine function is continuous at
\([0, \frac{\pi}{2}]\) and it assumes all values in \([0, 1]\), then there must
exist a point where \(\sin x = f'(c)\) by intermediate value theorem.

\subsection{Problem 4}
\label{sec:orgheadline10}
\begin{enumerate}
\item Prove that \(\sin x + \cos x \geq 1\) in the interval \([0, \frac{\pi}{2}]\),
as well as in \([2\pi k, 2\pi k + \frac{\pi}{2}]\), where \(k\) is a natural
number.
\item Prove that \(f(x) = e^x \sin x\) isn't uniformly differentiable on \([0, \infty)\).
\item Prove that \(f(x) = e^x \sin x\) is uniformly differentiable on \((-\infty, 0]\).
\end{enumerate}

\subsubsection{Answer 4}
\label{sec:orgheadline7}
The proof is immediate from \(\sin^2 x + \cos^2 x = 1\). Since \(\sin x \leq 1\)
and \(\cos x \leq 1\), it follows that \(\sin x \geq \sin^2 x\), and similarly
for cosine whenever both of them are positive.  Hence \(\sin x + \cos x \geq
    1\).  Since both sine and cosine are periodical with the period equal to
\(2\pi\), i.e. \(\sin x = \sin 2k\pi x\), it follows that \(\sin 2k\pi x + \cos
    2k\pi x \geq 1\) for any natural \(k\).

\subsubsection{Answer 5}
\label{sec:orgheadline8}

\subsubsection{Answer 6}
\label{sec:orgheadline9}

\subsection{Problem 5}
\label{sec:orgheadline12}
Let \(f\) be differentiable twice in interval \((a, b)\).  Let its second derivative
be strictly positive.  Prove that for every two points in the interval it holds
that \(f(x) \geq f(y) + f'(y)(x - y)\).

\subsubsection{Answer 7}
\label{sec:orgheadline11}

\subsection{Problem 6}
\label{sec:orgheadline14}
Let \(f\) be continuous in \(\mathbb{R}\).  Prove that between every two local maxima
of \(f\) there exists a local minimum point.

\subsubsection{Answer 8}
\label{sec:orgheadline13}
\end{document}