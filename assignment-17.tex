% Created 2015-06-20 Sat 23:26
\documentclass[a4paper]{article}
\usepackage[utf8]{inputenc}
\usepackage[T1]{fontenc}
\usepackage{fixltx2e}
\usepackage{graphicx}
\usepackage{longtable}
\usepackage{float}
\usepackage{wrapfig}
\usepackage{rotating}
\usepackage[normalem]{ulem}
\usepackage{amsmath}
\usepackage{textcomp}
\usepackage{marvosym}
\usepackage{wasysym}
\usepackage{amssymb}
\usepackage{capt-of}
\usepackage{hyperref}
\tolerance=1000
\usepackage[utf8]{inputenc}
\usepackage[usenames,dvipsnames]{color}
\usepackage{commath}
\usepackage{tikz}
\usetikzlibrary{shapes,backgrounds}
\usepackage{marginnote}
\usepackage{listings}
\usepackage{color}
\usepackage{enumerate}
\hypersetup{urlcolor=blue}
\hypersetup{colorlinks,urlcolor=blue}
\setlength{\parskip}{16pt plus 2pt minus 2pt}
\definecolor{codebg}{rgb}{0.96,0.99,0.8}
\definecolor{codestr}{rgb}{0.46,0.09,0.2}
\DeclareMathOperator{\Dom}{Dom}
\allowdisplaybreaks[4]
\author{Oleg Sivokon}
\date{\textit{<2015-06-19 Fri>}}
\title{Assignment 17, Infinitesimal Calculus}
\hypersetup{
 pdfauthor={Oleg Sivokon},
 pdftitle={Assignment 17, Infinitesimal Calculus},
 pdfkeywords={Infinitesimal Calculus, Assignment, Limits of functions},
 pdfsubject={Fourth asssignment in the course Infinitesimal Calculus},
 pdfcreator={Emacs 25.0.50.1 (Org mode 8.3beta)}, 
 pdflang={English}}
\begin{document}

\maketitle
\tableofcontents

\definecolor{codebg}{rgb}{0.96,0.99,0.8}
\lstnewenvironment{maxima}{%
  \lstset{backgroundcolor=\color{codebg},
    escapeinside={(*@}{@*)},
    aboveskip=20pt,
    showstringspaces=false,
    frame=single,
    framerule=0pt,
    basicstyle=\ttfamily\scriptsize,
    columns=fixed}}{}
}
\makeatletter
\newcommand{\verbatimfont}[1]{\renewcommand{\verbatim@font}{\ttfamily#1}}
\makeatother
\verbatimfont{\small}%
\clearpage

\section{Problems}
\label{sec:orgheadline17}

\subsection{Problem 1}
\label{sec:orgheadline3}
\begin{enumerate}
\item Compute
\begin{align*}
  \lim_{n \to \infty}\left(1 + \sqrt{n}\sin \frac{1}{n}\right)^{\sqrt{n}}
\end{align*}

\item Compute
\begin{align*}
  \lim_{x \to 0}\left(1 + \frac{1 - \cos x}{x}\right)^{\frac{1}{x}}
\end{align*}
\end{enumerate}

\subsubsection{Answer 1}
\label{sec:orgheadline1}
At first step we can replace \(m = \sqrt{n}\).  This gives us:
\begin{align*}
  \lim_{n \to \infty}\left(1 + m\sin(m^{-2})\right)^m
\end{align*}

Now, notice that \(\lim_{n \to \infty} m = \infty\), and that the formula
obtained is very similar to \(\lim_{b \to \infty}(1 + a)^b = e\), whre \(a\)
tends to zero.  Now, if \(m\sin(m^{-2})\) tends to zero, we know the limit to
be \(e\).  So, all we need to show is that \(\lim_{m \to \infty}m\sin(m^{-2}) =
    0\).
\begin{align*}
  &\textit{Define support variable k} \\
  &k = \frac{1}{m} \\
  &\lim_{m \to \infty}m\sin(m^{-2}) = \\
  &\lim_{k \to 0}\frac{\sin k^2}{k} = \\
  &\textit{Using L'Hospital's rule} \\
  &\lim_{k \to 0}\frac{(-\sin(k^2)) \cdot 2k}{1} = \\
  &\textit{Sinus is defined and is continuous at 0, simply substitute} \\
  &\lim_{k \to 0}\frac{0 \cdot 2 \cdot 0}{1} = 0\;.
\end{align*}

Since we showed \(\lim_{m \to \infty}m\sin(m^{-2}) = 0\), it follows that
\(\lim_{n \to \infty}\left(1 +
    \sqrt{n}\sin\left(\frac{1}{n}\right)\right)^{\sqrt{n}} = e\).

\subsubsection{Answer 2}
\label{sec:orgheadline2}
Similar to the previous answer, we will at first define a helper variable:
\(y = \frac{1}{x}\), then we will search for the solution of equivalent problem:
\begin{align*}
  &\lim_{y \to \infty} \left(1 + \frac{1 - \cos\left(\frac{1}{y}\right)}{\frac{1}{y}}\right)^y = \\
  &\lim_{y \to \infty} \left(1 + y\left(1 - \cos\left(\frac{1}{y}\right)\right)\right)^y = \\
  &\lim_{y \to \infty} \left(1 + 
    y\left(1 - \cos\left(\frac{1}{y}\right)\right)\right)^{
    \frac{1}{y(1 - \cos(\frac{1}{y}))}\cdot y^2(1 - \cos(\frac{1}{y}))} = \\
  &\lim_{y \to \infty}e^{y^2(1 - \cos(\frac{1}{y}))}
\end{align*}

Now, since the limit of the exponent is the exponent of the limits, we may
limit ourselves to finding the limit of \(y^2(1 - \cos(\frac{1}{y}))\).
Again, define a helper variable \(z = \frac{1}{y}\) and search for the limit as \(z\)
approaches zero:
\begin{align*}
  &\lim_{z \to 0} \frac{1 - \cos(z)}{z^2} = \\
  &\textit{Using L'Hospital} \\
  &\lim_{z \to 0} \frac{\sin(z)}{2z} = \\
  &\textit{Using L'Hospital again} \\
  &\lim_{z \to 0} \frac{\cos(z)}{2} = \\
  &\textit{Cosine is defined at 0 and is continuous, substituting} \\
  &\lim_{z \to 0} \frac{1}{2} = \frac{1}{2}\;.
\end{align*}

Substituting the intermediate result back gives: \(e^{\frac{1}{2}} = \sqrt{e}\).

\subsection{Problem 2}
\label{sec:orgheadline7}
Let \(f(x) = e^{-x} + \cos x\).
\begin{enumerate}
\item Prove \(\lim_{n \to \infty}f(\pi + 2\pi n) = -1\).
\item Prove \(\inf f([0, \infty)) = -1\).
\item Prove that for all \(-1 < c < 2\) there exists a solution for \(f(x) = c\) in
\([0, \infty)\).
\end{enumerate}

\subsubsection{Answer 3}
\label{sec:orgheadline4}
Limit of the sum is the sum of the limits, hence:
\begin{align*}
  &\lim_{x \to 0} e^{-x} + \cos x = \\
  & \lim_{x \to 0} e^{-x} + \lim_{x \to 0} \cos x = \\
  &\textit{Regardless of how we sample x, since $e^{-x}$ is monotonicaly decreasing} \\
  &0 + \lim_{x \to 0} \cos x = \\
  &\textit{Since cosine is continuous everywhere substituting} \\
  &\lim_{x \to 0} \cos(\pi + 2\pi n) = -1\;.
\end{align*}

\subsubsection{Answer 4}
\label{sec:orgheadline5}
The proof amounts to showing that \(e^{-x}\) is monotonically decreasing, and
has its infimum at zero.  Since infimum of cosine is at \(-1\), this would complete
the proof.  (Recall that the sum of infima of two functions is no less then
their sum.)  To show that \(e^{-x}\) is monotonically decreasing, and thus must
be bounded from below by its limit we claim that for any \(x_0 < x_1\) it is also
the case that \(e^{-x_0} > e^{-x_1}\) (the clame we made without a proof in the
previous answer.)  But this is immediate from definition of exponentiation.
Therfore the proof is complete.

\subsubsection{Answer 5}
\label{sec:orgheadline6}
The sum of two continuous functions is continuous, therefore \(f\) is
continuous in the same range where \(e^{-x}\) and cosine are continuous, and
in particular in the range \((-1, 2)\).  By intermediate value theorem, we are
guaranteed that \(f\) attains the value \(c\) in the specified range if we can
show that it is defined at the edges.  In the previous answers we found that
\(f\) has a limit point at \(-1\), in other words, we can make it as close to
\(-1\) as we like.  Solving for \(c = 2\) is tricky, but we can pick a larger
value, without harming the claim, for example, pick \(x = -\frac{\pi}{2}\).
This gives \(e^{\frac{\pi}{2}} + \cos(\frac{\pi}{2}) = e^{\frac{\pi}{2}} + 0
    > 4 > 2\).

\subsection{Problem 3}
\label{sec:orgheadline10}
\begin{enumerate}
\item Prove \(\lim_{n \to \infty}(\ln(2\pi n + \frac{\pi}{2}) - \ln(2\pi n)) = 0\).
\item Prove that \(f(x) = \sin(e^x)\) is not uniformly continuous.
\end{enumerate}

\subsubsection{Answer 6}
\label{sec:orgheadline8}
Using the properties of \(ln\), viz. \(ln(x) - ln(y) = ln(x / y)\) obtains:
\begin{align*}
  &\lim_{n \to \infty} \left(\ln(2\pi n + \frac{\pi}{2}) - \ln(2\pi n)\right) = \\
  &\lim_{n \to \infty} \ln\left(\frac{2\pi n + \frac{\pi}{2}}{2\pi n}\right) = \\
  &\textit{Using limit of function composition} \\
  &\ln\left(\lim_{n \to \infty} \frac{2\pi n + \frac{\pi}{2}}{2\pi n}\right) = \\
  &\textit{Using L'Hospital's rule} \\
  &\ln\left(\lim_{n \to \infty} \frac{2\pi}{2\pi}\right) = \\
  &\ln(1) = 0\;.
\end{align*}

\subsubsection{Answer 7}
\label{sec:orgheadline9}
We are going to use the definition of uniform continuity which requires that
if a limit of a difference of two sequences is equal to zero, then the limit
of the difference of sequences of function's values at these sequences must
be zero too.

Let \((x_n) = \ln(n + 2)\) and \((y_n) = \ln(n)\).  The proof of the limit of
their difference being equal to zero is identical to the one given in the
previous answer.

Now consider these two sequences \((x_{f(n)}) = e^{\ln(n + 2)}\) and
\((y_{f(n)}) = e^{\ln(n)}\).  From definition of uniform continuity, it
follows that:

\(\lim_{n \to \infty}\left(e^{\ln(n + 2)} - e^{\ln(n)}\right) = 0\) too.

\begin{align*}
  &\lim_{n \to \infty}\left(e^{\ln(n + 2)} - e^{\ln(n)}\right) = \\
  &\textit{Since $e^{\ln(x)} = x$} \\
  &\lim_{n \to \infty}\left(n + 2 - n\right) = 2\;.
\end{align*}

Contrary to assumed.  Hence \(f(x) = e^x\) is not uniformly continuous.

\subsection{Problem 4}
\label{sec:orgheadline11}
For all functions given below find their domain of definition, domain of
continuity, and domain of differentiability.  Find a definition for every
point in the differentiability domain \emph{(I have no idea what this is supposed
to mean)}.
\begin{enumerate}
\item \begin{align*}
  f(x) = \begin{cases}
    x^2 \sin \frac{1}{x^2} &\mbox{if} x \neq 0 \\
    0                      &\mbox{if} x = 0\;.
  \end{cases}
\end{align*}

\item \(f'(x)\) for \(f\) defined in previous question.
\item \begin{align*}
  f(x) = x^2D(x) = \begin{cases}
    x^2 &\mbox{if} x \in \mathbb{Q} \\
    0   &\mbox{if} x \not \in \mathbb{Q}\;.
  \end{cases}
\end{align*}
\end{enumerate}

\subsection{Problem 5}
\label{sec:orgheadline14}
Prove that:
\begin{enumerate}
\item \begin{align*}
  \lim_{x \to 0^-}\frac{e^{\frac{1}{x}}}{x} = 0\;.
\end{align*}

\item Prove that the function:
\begin{align*}
  f(x) = \begin{cases}
    e^{\frac{1}{x}} + \sin x &\mbox{if } x < 0 \\
    \ln(1 + x)             &\mbox{if } x \geq 0
  \end{cases}
\end{align*}

is differentiable at 0.
\end{enumerate}

\subsubsection{Answer 8}
\label{sec:orgheadline12}
Define helper variable \(y = \frac{1}{x}\), then we need to solve an equivalent
problem: \(\lim_{y \to \infty}\frac{1}{ye^{y}}\).  But the solution is immediate
since \(e^n \geq 1\) whenever \(n > 0\).  In other words, this is:
\(\lim_{y \to \infty}\frac{1}{y \cdot 1} = 0\).

\subsubsection{Answer 9}
\label{sec:orgheadline13}
To show that \(f\) is differentiable we need to show that both limits exist and
that they agree at 0, furthermore, that the limits are finite.  In other words:
\begin{align*}
  &\lim_{x \to 0^-}\left(e^{\frac{1}{x}} + \sin x\right) = \lim_{x \to 0^+}\ln(1 + x) \\
  &\textit{Logarithm is continuous at 1, substituting value for x:} \\
  &\lim_{x \to 0^-}\left(e^{\frac{1}{x}} + \sin x\right) = 0 \\
  &\lim_{x \to 0^-}\left(e^{\frac{1}{x}}\right) + \lim_{x \to 0^-}\left(\sin x\right) = 0 \\
  &\textit{Sine is continuous at 0, substituting value for x:} \\
  &\lim_{x \to 0^-}\left(e^{\frac{1}{x}}\right) + 0 = 0\;.
\end{align*}

In order to find the later limit, we could define a helper variable: \(y = \frac{1}{x}\)
and solve an equivalent problem:
\begin{align*}
  &\lim_{y \to \infty}\left(\frac{1}{e^y}\right) = 0
\end{align*}

Since \(e^y\) is monotonically increasing and has no upper bound.

Having showed that, we showed that both left and right limits exist and that they
agree at 0, hence \(f\) is differentiable at 0.

\subsection{Problem 6}
\label{sec:orgheadline16}
Let \(f\) be continuous at \(x_0\).  Prove that \(g(x) = \abs{x}f(x)\) is
differentiable if and only if \(f(0) = 0\).


\subsubsection{Answer 10}
\label{sec:orgheadline15}
The intuition for the proof is that additive identity in an ordered field is
unique in that it is the only positive number, which doesn't preserve sign
under multiplication.  In other words, since the derivatives of absolute value
function for number less than zero and for number greater than zero differ
only in sign, the only way to ignore this discrepancy is to multiply them
by zero.  A more formal proof follows:

First, we will prove the ``if'' part, i.e. if \(f(0) = 0\), then \(g\) is
differentiable.
\begin{align*}
  &\lim_{h \to 0}\frac{g(0 + h) - g(0)}{h}\cdot f(h) \;\textit{where}\; g(h) = \abs{h} \\
  &\lim_{h \to 0}\frac{\abs{0 + h} - \abs{0}}{h} \cdot g(h) = 
  \lim_{h \to 0}\frac{\abs{h}}{h} \cdot g(h) \\
  &\textit{Assume}\; h > 0 \;\textit{then} \\
  &\lim_{h \to 0^+}\frac{\abs{h}}{h} = 1 \cdot g(h) \\
  &\textit{Assume}\; f(h) = 0 \;\textit{then} \\
  &\lim_{h \to 0^+} 1 \cdot 0 = 0 \\
  &\textit{else} \\
  &\lim_{h \to 0^-}\frac{\abs{h}}{h} = -1 \\
  &\textit{Assume}\; f(h) = 0\; \textit{then, similarly} \\
  &\lim_{h \to 0^-} -1 \cdot 0 = 0 \\
  &0 = 0\;.
\end{align*}

Since both limits exists and agree, \(g\) is differentiable.

Now, we will prove the ``only if'' part.  Suppose, for contradiction there
existed some value, we define it later to be \(y \neq 0\), exists, such that
\(g\) would be differentiable at 0:
\begin{align*}
  &\textit{Assume}\; \lim_{h \to 0^+} \frac{\abs{h}}{h} \cdot f(0) =
  \lim_{h \to 0^-} \frac{\abs{h}}{h} \cdot f(0) \\
  &\lim_{h \to 0^+} 1 \cdot f(0) = \lim_{h \to 0^-} -1 \cdot f(0) \\
  &\lim_{h \to 0^+} f(0) = -\lim_{h \to 0^-} f(0) \\
  &\textit{Since $f$ is continuous, its value at 0 is its limit at 0} \\
  &\textit{Put $f(0) = y, y \neq 0$ then}\\
  &-y = y\; \textit{Contradiction!}
\end{align*}

Even though both limits exist, they don't agree, hence whenever \(g\) is not
differentiable, \(f(0) \neq 0\).
\end{document}